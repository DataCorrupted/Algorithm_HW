 
\documentclass{article}

\title{Algorithm Homework 5}
\author{Rong Yuyang \\ Student ID: 69850764 \\ rongyy@shanghaitech.edu.cn}

\usepackage[utf8]{inputenc}
\usepackage{graphicx}
\usepackage[colorlinks,linkcolor=red]{hyperref}
\usepackage{amsmath, amsthm, amssymb}
\usepackage{subfloat}
\newtheorem{prop}{Proposition}
\usepackage{ulem}
\usepackage{indentfirst}
\usepackage{listings}
\usepackage{color}
\usepackage{amsmath}

\definecolor{codegreen}{rgb}{0,0.6,0}
\definecolor{codegray}{rgb}{0.5,0.5,0.5}
\definecolor{codepurple}{rgb}{0.58,0,0.82}
\definecolor{backcolour}{rgb}{0.95,0.95,0.92}


\lstdefinestyle{mystyle}{
    backgroundcolor=\color{backcolour},   
    commentstyle=\color{codegreen},
    keywordstyle=\color{magenta},
    numberstyle=\tiny\color{codegray},
    stringstyle=\color{codepurple},
    basicstyle=\footnotesize,
    breakatwhitespace=false,         
    breaklines=true,                 
    captionpos=b,                    
    keepspaces=true,                 
    numbers=left,                    
    numbersep=5pt,                  
    showspaces=false,                
    showstringspaces=false,
    showtabs=false,                  
    tabsize=2
}
\lstset{style=mystyle}
\begin{document}
\maketitle

\section*{Problem 5-1}
\subsection*{(a)}
\begin{lstlisting}[language = python]
def MinCost( V, E, P):
	T = MinSpanTree(V, E)
	max_p = max(T)
	min_c = inf;
	for (c, p) in P:
		if p>=max_p and c<min_c:
			min_c = c;
# If return value is inf, then there is no such a plan.
return min_c
\end{lstlisting}
\section*{Problem 5-2}
\subsection*{(a)}
\begin{lstlisting}[language = python]
def bfs(V, E):
# V: a list of all nodes. E: a dictionary of all edges.
    queue = [V[1]];
    H = {V[1] = 1}
    i = 0; j = 0;
    while i <= j:
        curr = queue[i]; i += 1;
        # Edge exist and not visited.
        while (v, curr) in E and (not v in H):
            if E[(curr, v)] % 1 == 1:
            # That the edge is odd, the node can't have the same value.
                H[v] = H[curr]
            else
                H[v] = 1 - H[curr]
            j += 1;
            queue[j] = v
    return H
\end{lstlisting}
\subsection*{(b)}.\\
\textbf{Claim 1} Such $h$ always exists for a tree.
\begin{proof}
	Proof by induction.\\
	1 node tree $T_1$: $h(v) = 1$. proved.\\
	n node tree $T_n$: suppose there is a function $h$.\\
	n+1 node tree $T_{n+1}$: take a leaf node $v$. We can form a $h$ without $v$ since $T_{n+1}/v = T_{n}$. Then we extend $h$: \\
		if $(v, v.parent)$ is odd, $h(v) = h(v.parent)$, else $h(v) = 1 - h(v.parent)$
\end{proof}.\\
\textbf{Claim 2a}: If $h(s) \neq h(t)$, any path from $s$ to $t$ contains odd numbers of even edges.\\
\textbf{Claim 2b}: If $h(s) = h(t)$, any path from $s$ to $t$ contains even numbers of even edges.
\begin{proof}
	Proof by induction \\
	Path with 1 edge. \\
	$h(s) \neq h(t)$, the edge must be even. One(odd) even edge present. \\
	$h(s)   =  h(t)$, the edge must be odd. No(even) even edge present. \\
	Path with n edges. Suppose this is true.
	Path n+1 edges. \\
	take $t$ out and examine $t.parent$. The path from $s$ to $t.parent$ has n edges.\\
	Claim 2a: When $h(s) \neq h(t)$: \\
	$h(s) = h(t.parent) \neq h(t)$, Even + 1(Odd) even edges present.\\
	$h(s) = h(t) \neq h(t.parent)$, Odd + 0(Odd) even edges present.\\
	Claim 2b: When $h(s) = h(t)$: \\
	$h(s) = h(t.parent) = h(t)$, Even + 0(Even) even edges present.\\
	$h(s) = h(t) \neq h(t.parent) $, Odd + 1(Even) even edges present.
\end{proof}
Now let's proof that $\exists h$ when $G$ has no odd number of even weight edge.
\begin{proof}
	Suppose $G = (V, E) = (V, T + S)$ where $(V, T)$ is a span tree of $G$. According to claim 1 $\exists h$ for span tree $(V, T)$. All we need to do is to proof that this $h$ is valid for $G$. To be more precise, $h$ is valid $\forall s = (u, v) \in S$\\
	Case 1: if $h(u) = h(v)$ and $s$ is odd, then $h$ is still valid. \\
	Case 2: if $h(u) \neq h(v)$ and $s$ is even, then $h$ is still valid. \\
	Case 3: if $h(u) = h(v)$ and $s$ is even, then a path from $u$ to $v$ already have even number of even edges, adding one more even edge $s$ will lead to a circle with odd number of even edges, which is not possible. \\
	Case 4: if $h(u) \neq h(v)$ and $s$ is odd, then a path from $u$ to $v$ already have odd number of even edges, with $s$ being odd we have a circle with odd number of even edges, which, again, is not possible. \\

\end{proof}
\section*{Problem 5-3}
\begin{lstlisting}[language = python]
def bfs(s, V , E)
    queue = s
    dist = {s: 0}
    i = 0; j = 0;
    while i <= j:
        curr = queue[i]; i += 1;
        while (v, curr) in E and (not v in dist):
            dist[v] = dist[curr] + E[(curr, v)]
            j += 1; queue[j] = v;
    reutrn (max(dist), argmax(dist))

(_, s) = bfs(1, V, E)
(ans, _) = bfs(s, V, E)
\end{lstlisting}
\par \textbf{Time Complexity} DFS on a graph runs on $O(|E| + |V|)$, since in this case the graph is a tree: $O(|E|) = O(|V|) = O(n)$. Thus two times dfs runs in:
$$T(n) = O(n) = O(n^2) = O(n^3)$$
\par \textbf{Proof of correctness} 
\par \textbf{Claim} assume path $ P = \{s, \cdots, v_i, \cdots, t\}$ is the longest path in the tree, a dfs for longest path starting from $\forall v$ in the tree ends at node $v_e$. We have $v_e \in \{s, t\}$
\begin{proof}. \\
    \textbf{Case 1}: $v \in P$, then this claim is true. 
    \par Suppose not, $d(u, v_e) > d(u, t), d(u, v_e) > d(u, s)$, then $d(s, t)$ is not the longest path, contradiction. \\
    \textbf{Case 2}: $v \notin P$ \\
    	Case I, from $u$ to $v_e$ there is a crossing $x \in P$ with $s$ to $t$, then we fall into Case i.
    	Case II, $\{u \cdots v_e \}$ is disjoint with $P$, then $dis(u, v_e) > dis(u, x) + dis(x, t)$ where $x \in P$, contradiction.
    Thus, the claim is true.
\end{proof}
In This case, the ending point of the first bfs must be the starting node of the longest path, all we need is another bfs.
\end{document}
